\documentclass{bioinfo}
\usepackage{url}
\copyrightyear{2011}
\pubyear{2011}

\begin{document}
\firstpage{1}

\title[Parallel Motif Finding]{Search for Superlinear Parallelism in Motif Finding Using Software Transactional Memory}
\author[D. G. Wilcox]{David G. Wilcox}
\address{University of Sydney, Student Department}

\history{Received on 2011/10/15; revised on 2011/10/15; accepted on 2011/10/15}

\editor{Editor: D. G. Wilcox}

\maketitle

\begin{abstract}

\section{Motivation:}
It is not necessary to argue the importance of parallel programming for the future of computer science. That said, the phenomenon known as super linear speedup, where the speed of the program is increased by more than the number of additional cpu cores it runs on, was the icing on the cake. Additionally there was a large focus on scaling the simplicity of the code inspite of the sophisticated concepts used.

\section{Results:}
For the test data used, and the combinations of fork/core, superlinear speedup was detected in a "sweetspot" when using 4-8 forks/core and 8-10 cores total. Other combinations either fell in the range of providing a sub-linear speedup or a loss in speed due to overhead costs. Typically using more cores or forks/core than the sweetspot resulted in a loss to overhead, and less was a sub-linear speedup. Although the observations made are not in doubt, there is a significant lack in precision, obstructing a deeper investigation into the phenomenon.

\section{Availability:}
The implementation was in Haskell, which possesses many free and open source compilers (GHC with the Haskell Platform is the suggested). Testing scripts and benchmarking was done in Ruby, also free and open source. The entirety of the project is hosted on github at \url{https://github.com/goldfish-dave/BioinformaticsAssignment}.

\section{Contact:} \href{dwil5124@uni.sydney.edu.au}{dwil5124@uni.sydney.edu.au}
\end{abstract}

\section{Introduction}

There were three main goals in writing this paper. They were exploiting multiple cores to get a speed increase, choosing an interesting example where superlinear speedup(SLS) could be observed and not sacrificing elegance, composability or simplicity in achieving these goals.

Before I go any further it is important to define SLS. In a parallel program, if we split the work evenly over all the available cores then, ignoring overhead costs in organising the process, the most we could expect of our program is that for $n$ cores we get a speedup factor of $n$ (e.g. 2 cores results in half the runtime). In this event we would have an efficiency of $100\%$. A SLS is defined as having an efficiency greater than $100\%$.

How can this be possible? The first thing to admit is that it is not always possible; there is a dependance on the way the data is structued. That is, the actions of one core must have a positive effect (decrease in runtime) on the actions of another core. The simplest example of this is in a branch and bound tree search. If one core finds a superior cut-off value and shares it with the other cores, then all cores will benefit from the actions of the one.

It is for this reason that the motif finding algorithm, Median String Search, was chosen for investigation.

As an aside, there are other possible causes for a SLS involving tiered memory, but that hardware related phenomenon will not be investigated in this software based report.

Text Text Text Text Text Text  Text Text Text Text Text Text Text Text Text  Text Text Text Text Text Text. Figure~\ref{fig:01} shows that the above method  Text Text Text Text  Text Text Text Text Text Text  Text Text.  \citep{Bag01} wants to know about �� text follows.

\section{Approach}

In order to meet the above requirement for producing a SLS, a concurrent branch and bound algorithm must be implemented. Unfortunately this rules out classes of parallel programming styles involving implicit parallelism of deterministic computations using strategies. After initial research a cluster programming approach was discarded in favour of the more supported multicore approach. Which raises the common multicore question: if one core reads the global cut-off, finds a better cut-off, and goes to update the global cut-off, how do we deal with the case where the global cut-off has \textit{already} been updated with \textit{an even better} cut-off? Thankfully the finally result won't change, but it may take longer to compute now. Do we take no extra action and allow our recently improved cut-off to be reverted to an inferior cut-off improvement? Or do we, somehow, prevent this from happening? The somehow is likely to be expensive, either computationally (interfering with our search for SLS) or conceptually (interfering with our attempt to control complexity). Both approaches were implemented.

That said, due to the way in which forking was implemented and controlled (a global list of live forks), race conditions were encountered and tackled in both implementations.

The question remains then, how do we prevent race conditions? The conventional approach is using locks on the resource throughout the process, which has it's strengths and weaknesses, but the approach I decided on was Software Transactional Memory (STM). The key reason was this: interesting to read about it may be, the finer details of the process used (rollback and retry) are abstracted neatly, allowing simple and composable code hopefulyl without sacrificing too much performance.

Equation~(\ref{eq:01}) Text Text Text Text Text Text  Text Text Text Text Text Text Text Text Text  Text Text Text Text Text Text. Figure \ref{fig:02} shows that the above method  Text Text Text Text  Text Text Text Text Text Text  Text Text.  \citealp{Boffelli03} might want to know about  text text text text ��


\begin{methods}
\section{Methods}



\end{methods}

\begin{figure}[!tpb]%figure1
%\centerline{\includegraphics{fig01.eps}}
\caption{Caption, caption.}\label{fig:01}
\end{figure}

\begin{figure}[!tpb]%figure2
%\centerline{\includegraphics{fig02.eps}}
\caption{Caption, caption.}\label{fig:02}
\end{figure}

\section{Discussion}

Text Text Text Text Text Text  Text Text Text Text Text Text Text Text Text  Text Text Text Text Text Text. Figure \ref{fig:02} shows that the above method  Text Text Text Text  Text Text Text Text Text Text  Text Text.  \citealp{Boffelli03} might want to know about  text text text text
Text Text Text Text Text Text  Text Text Text Text Text Text Text Text Text  Text Text Text Text Text Text. Figure \ref{fig:02} shows that the above method  Text Text Text Text  Text Text Text Text Text Text  Text Text.  \citealp{Boffelli03} might want to know about  text text text text
Text Text Text Text Text Text  Text Text Text Text.




Table~\ref{Tab:01} shows that Text Text Text Text Text  Text Text Text Text Text Text. Figure \ref{fig:02} shows that
the above method Text Text. Text Text Text  Text Text Text Text Text Text. Figure \ref{fig:02} shows that
the above method Text Text. Text Text Text  Text Text Text Text Text Text. Figure \ref{fig:02} shows that
the above method Text Text.









%%%%%%%%%%%%%%%%%%%%%%%%%%%%%%%%%%%%%%%%%%%%%%%%%%%%%%%%%%%%%%%%%%%%%%%%%%%%%%%%%%%%%
%
%     please remove the " % " symbol from \centerline{\includegraphics{fig01.eps}}
%     as it may ignore the figures.
%
%%%%%%%%%%%%%%%%%%%%%%%%%%%%%%%%%%%%%%%%%%%%%%%%%%%%%%%%%%%%%%%%%%%%%%%%%%%%%%%%%%%%%%






\section{Conclusion}

(Table~\ref{Tab:01}) Text Text Text Text Text Text  Text Text Text Text Text Text Text Text Text  Text Text Text Text Text Text. Figure \ref{fig:02} shows that the above method  Text Text Text Text  Text Text Text Text Text Text  Text Text.  \citealp{Boffelli03} might want to know about  text text text text
Text Text Text Text Text Text  Text Text Text Text Text Text Text Text Text  Text Text Text Text Text Text. Figure \ref{fig:02} shows that the above method  Text Text Text Text  Text Text Text Text Text Text  Text Text.  \citealp{Boffelli03} might want to know about  text text text text
Text Text Text Text Text Text  Text Text Text Text Text Text Text Text Text  Text Text Text Text Text Text. Figure \ref{fig:02} shows that the above method  Text Text Text Text  Text Text Text Text Text Text  Text Text.



Text Text Text Text Text Text  Text Text Text Text Text Text Text Text Text  Text Text Text Text Text Text. Figure \ref{fig:02} shows that the above method  Text Text Text Text  Text Text Text Text Text Text  Text Text.  \citealp{Boffelli03} might want to know about  text text text text





\begin{enumerate}
\item this is item, use enumerate
\item this is item, use enumerate
\item this is item, use enumerate
\end{enumerate}

Text Text Text Text Text Text  Text Text Text Text Text Text Text Text Text  Text Text Text Text Text Text. Figure \ref{fig:02} shows that the above method  Text Text Text Text  Text Text Text Text Text Text  Text Text.  \citealp{Boffelli03} might want to know about  text text text text
Text Text Text Text Text Text  Text Text Text Text Text Text Text Text Text  Text Text Text Text Text Text. Figure \ref{fig:02} shows that the above method  Text Text Text Text  Text Text Text Text Text Text  Text Text.  \citealp{Boffelli03} might want to know about  text text text text
Text Text Text Text Text Text  Text Text Text Text Text Text Text Text Text  Text Text Text Text Text Text.






Text Text Text Text Text Text  Text Text Text Text Text Text Text Text Text  Text Text Text Text Text Text. Figure \ref{fig:02} shows that the above method  Text Text Text Text


\section*{Acknowledgement}
Text Text Text Text Text Text  Text Text.  \citealp{Boffelli03} might want to know about  text text text text

\paragraph{Funding\textcolon} Text Text Text Text Text Text  Text Text.

%\bibliographystyle{natbib}
%\bibliographystyle{achemnat}
%\bibliographystyle{plainnat}
%\bibliographystyle{abbrv}
%\bibliographystyle{bioinformatics}
%
%\bibliographystyle{plain}
%
%\bibliography{Document}


\begin{thebibliography}{}
\bibitem[Bofelli {\it et~al}., 2000]{Boffelli03} Bofelli,F., Name2, Name3 (2003) Article title, {\it Journal Name}, {\bf 199}, 133-154.

\bibitem[Bag {\it et~al}., 2001]{Bag01} Bag,M., Name2, Name3 (2001) Article title, {\it Journal Name}, {\bf 99}, 33-54.

\bibitem[Yoo \textit{et~al}., 2003]{Yoo03}
Yoo,M.S. \textit{et~al}. (2003) Oxidative stress regulated genes
in nigral dopaminergic neurnol cell: correlation with the known
pathology in Parkinson's disease. \textit{Brain Res. Mol. Brain
Res.}, \textbf{110}(Suppl. 1), 76--84.

\bibitem[Lehmann, 1986]{Leh86}
Lehmann,E.L. (1986) Chapter title. \textit{Book Title}. Vol.~1, 2nd edn. Springer-Verlag, New York.

\bibitem[Crenshaw and Jones, 2003]{Cre03}
Crenshaw, B.,III, and Jones, W.B.,Jr (2003) The future of clinical
cancer management: one tumor, one chip. \textit{Bioinformatics},
doi:10.1093/bioinformatics/btn000.

\bibitem[Auhtor \textit{et~al}. (2000)]{Aut00}
Auhtor,A.B. \textit{et~al}. (2000) Chapter title. In Smith, A.C.
(ed.), \textit{Book Title}, 2nd edn. Publisher, Location, Vol. 1, pp.
???--???.

\bibitem[Bardet, 1920]{Bar20}
Bardet, G. (1920) Sur un syndrome d'obesite infantile avec
polydactylie et retinite pigmentaire (contribution a l'etude des
formes cliniques de l'obesite hypophysaire). PhD Thesis, name of
institution, Paris, France.

\end{thebibliography}
\end{document}
